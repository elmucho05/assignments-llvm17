\documentclass{article}
\usepackage[a4paper, margin=1in]{geometry}
\usepackage{longtable}
\usepackage{graphicx}
\usepackage{listings}
\usepackage{array}
\usepackage{etoolbox}
\newcolumntype{P}[1]{>{\centering\arraybackslash}p{#1}}


\title{Assignment 2}
\author{Algert Mucaj, Simone Garzarelli}
\date{April 2024}

\begin{document}
	\maketitle
\begin{section}{Very Busy Expressions}
	
	% Define a larger width for the longtable
%	\setlength{\LTcapwidth}{\textwidth} % Set the caption width to text width
	%	\setlength{\LTleft}{0pt} % Left-align the longtable

\large % Set overall font size to large
\renewcommand{\arraystretch}{1.5} % Increase row height for better readability

	
\begin{longtable}{| c | c |}
	\hline
	\multicolumn{2}{|c|}{\textbf{Very Busy Expressions}} \\
	\hline
	Domain & Insiemi di espressioni \\
	\hline
	Direction & Backward \\
	  & IN[b] = $f_b (OUT[b])$ \\
	  & OUT[b] = \verb*|^| IN[successori(b)] \\
	\hline
	Transfer function &  $f_b(x) = Gen_b \cup (x - Kills_b)$\\
	\hline
	Meet Operation & $\cap$ \\
	\hline
	Boundary Condition & IN[exit] = $\emptyset$ \\
	\hline
	Initial interior points & In[b] = $\mu$ \\
	\hline
	% Add more rows as needed
\end{longtable}

\begin{subsection}{Scelte implementative}
	Un’espressione è \textbf{very busy} in un
	punto \textit{p} se, indipendentemente
	dal percorso preso da \textit{p},
	l’espressione viene usata prima
	che uno dei suoi operandi venga
	definito. \\ \\
	Trattandosi di insiemi di espressioni, possiamo intuire che la direzione di scorrimento è come quella di quando trattiamo la liveness analysis di una variabile.
	\\\\
	Utilizzando una direzione "Backward", l'analisi delle "Very Busy Expressions" determina l'insieme di espressioni che saranno "molto usate" prima di ogni assegnazione o modifica di un operando. Questo tipo di analisi aiuta a precalcolare l'uso delle espressioni all'indietro nel flusso di esecuzione del programma, partendo dai punti di uscita (exit) fino al punto di ingresso (entry).
	\\ \\
	Il bit vector, di conseguenza, sarà costituito dalle espressioni $b-a$ e $a-b$.\\
	
	\textbf{Come leggere la tabella} delle iterazioni:
	Ovunque si trovi nella tabella la coppia (1,1), significa che in quella determinata configurazione, i bit corrispondenti alle espressioni $b-a$ e $a-b$, sono alti.

%TODO
% aggiungere la tabella di gen e kills.
\pagebreak

\begin{longtable}{| c | c | c |}
	\hline
	\textbf{BB} & \textbf{Gen} & \textbf{Kill}\\
	\hline
	1 & $\emptyset$ & $\emptyset$\\
	2 & $\emptyset$&$\emptyset$\\
	3 & b-a &$\emptyset$\\
	4 & a-b&$\emptyset$ \\
	5 & b-a&$\emptyset$\\
	6 & $\emptyset$& b-a,a-b\\
	7 & a-b&$\emptyset$ \\
	8 & $\emptyset$&$\emptyset$\\
	\hline
	\caption{Tabella Gen e Kill}\\
\end{longtable}

\end{subsection}
\begin{subsection}{Iterazioni dell'algoritmo}
	
	\begin{longtable}{| c | c | c | c | c | c | c |}
		\hline
		\textbf{BB} & \multicolumn{2}{|c|}{\textbf{Iterazione 1}} & 	\multicolumn{2}{|c|}{\textbf{iterazione 2}} & 	\multicolumn{2}{|c|}{\textbf{iterazione 3}} \\
		\hline
		\space & in & out & in & out & in & out \\
		\hline
		BB1 & 1,1 & 1,1 & 1,0 & 1,0 & 1,0 & 1,0 \\
		\hline
		BB2 & 1,1 & 1,1 & 1,0 & 1,0 & 1,0 & 1,0 \\
		\hline
		BB3 & 1,1 & 1,1 & 1,1 & 0,1 & 1,1 & 0,1 \\
		\hline
		BB4 & 1,1 & $\emptyset$ & 0,1 & $\emptyset$ & 0,1 & $\emptyset$ \\
		\hline
		BB5 & 1,1 & 1,1 & 1,0 & 0,0 & 1,0 & 0,0 \\
		\hline
		BB6 & 1,1 & 1,1 & 0,0 & 0,1 & 0,0 & 0,1 \\
		\hline
		BB7 & 1,1 & $\emptyset$ & 0,1 & $\emptyset$ & 0,1 & $\emptyset$ \\
		\hline
		BB8 & $\emptyset$ & 0,0 & $\emptyset$ & $\emptyset$ & $\emptyset$ & $\emptyset$ \\
		\hline
	\end{longtable}
\end{subsection}
\end{section}
\pagebreak
\begin{section}{Dominator Analysis}
	In un CFG diciamo che un nodo \textit{X} \textbf{domina} un altro	nodo \textit{Y} se il nodo \textit{X}	\textit{appare} \textit{in ogni percorso	}del grafo che porta dal 	blocco ENTRY al blocco \textit{Y}
	\begin{itemize}
		\item Annotiamo ogni basic block Bi con un insieme DOM[Bi]
			\subitem $B_i \in DOM[B_j]$ se e solo se Bi domina Bj
		\item Per definizione, un nodo \textit{domina} \textbf{se stesso}
	\end{itemize}
\begin{longtable}{| c | c |}
	\hline
	\multicolumn{2}{|c|}{\textbf{Dominator Analysis}} \\
	\hline
	Domain & Insiemi di Basic Block\\
	\hline
	Direction & Forward \\
	& OUT[b] = $f_b (IN[b])$ \\
	& IN[b] = \verb*|^| OUT[predecessori(b)] \\
	\hline
	Transfer function &  $f_b(x) = B \cup x$\\
	\hline
	Meet Operation & $\cap$ \\
	\hline
	Boundary Condition & OUT[entry] = entry \\
	\hline
	Initial interior points & OUT[b$_i$] = $\mu$ \\
	\hline
\end{longtable}
Non troviamo funzioni Gen e Kill poiché stiamo lavorando direttamente con dei basic block. L'intersezione è l'operazione che rispecchia meglio questa proprietà di inclusione. Quando calcoliamo l'OUT di un basic block \textit{b}, consideriamo l'intersezione degli IN dei suoi predecessori. Questo significa che l'OUT[b] deve contenere solo quei blocchi che sono dominatori comuni di tutti i predecessori di b.
\newline\newline
L'intersezione riduce l'insieme di possibili dominatori, mantenendo solo quei blocchi che sono dominatori comuni a tutti i percorsi di ingresso nel blocco.

\begin{longtable}{| c | c | c | c | c | c | c |}
	\hline
	\textbf{BB} & \multicolumn{2}{|c|}{\textbf{Iterazione 1}} & 	\multicolumn{2}{|c|}{\textbf{iterazione 2}} & 	\multicolumn{2}{|c|}{\textbf{iterazione 3}} \\
	\hline
	\space & in & out & in & out & in & out \\
	\hline
	A & A & A & A & A & A & A\\
	\hline
	B & A & B & A & AB & A & AB \\
	\hline
	C & AC & C & A & AC & A & AC \\
	\hline
	D & AC & D & AC & ACD & AC & ACD \\
	\hline
	E & AC & E & AC & ACE & AC & ACE \\
	\hline
	F & $\emptyset$ & F & AC & ACF & AC & ACF \\
	\hline
	G & $\emptyset$ & G & A & AG & A & AG \\
	\hline
\end{longtable}
%B cup x , B(blocco stesso) unito con In 
\end{section}
\pagebreak
\begin{section}{Constant Propagation}
	L’obiettivo della constant propagation è quello di determinare in quali punti del programma le variabili hanno un valore costante. 
	\\
	L’informazione da calcolare per ogni nodo n del CFG è un insieme di coppie del tipo \textit{(variabile, valore costante)}. Se abbiamo la coppia (x, c) al nodo n, significa che x è garantito avere il valore c ogni volta che n viene raggiunto
	durante l’esecuzione del programma.
	
	
\begin{longtable}{| c | c |}
	\hline
	\multicolumn{2}{|c|}{\textbf{Constant Propagation}} \\
	\hline
	Domain & Insiemi di coppie \textit{(variabile, costante)} \\
	\hline
	Direction & Forward \\
	& OUT[b] = $f_b (IN[b])$ \\
	& IN[b] = \verb*|^| OUT[predecessori(b)] \\
	\hline
	Transfer function &  $f_b(x) = Gen_b \cup (x - Kills_b)$\\
	\hline
	Meet Operation & $\cap$ \\
	\hline
	Boundary Condition & OUT[entry] = $\emptyset$ \\
	\hline
	Initial interior points & OUT[$b_i$] = $\mu$ \\
	\hline
	% Add more rows as needed
\end{longtable}

	
\begin{longtable}{| c | c | c |}
	\hline
	\textbf{BB} & \textbf{Gen} & \textbf{Kill}\\
	\hline
	entry & $\emptyset$ & $\emptyset$\\
	1 & (k,2) & $\emptyset$\\
	2 & $\emptyset$&$\emptyset$\\
	3 & (a,k+2) & $\emptyset$\\
	4 & (x,5)&$\emptyset$ \\
	5 & (a,4)&(a,k+2)\\
	6 & (x,8)& (x,5)\\
	7 & (k,a)&(k,2)\\
	8 & $\emptyset$&$\emptyset$\\
	9 & (b,2)& $\emptyset$\\
	10 & (x,a+k)& (x,8)(x,5)\\
	11 & (y,a*b)& $\emptyset$\\
	12 & (k, k+1) & (k,a)\\
	13 & $\emptyset$ & $\emptyset$\\
	exit & $\emptyset$ & $\emptyset$ \\
	\hline
	\caption{Tabella Gen e Kill}\\
\end{longtable}	
 Nella propagation delle costanti, l'obiettivo principale è determinare quali valori costanti possono essere conosciuti prima di eseguire ogni blocco di base del programma.L'analisi forward si allinea naturalmente con il flusso di esecuzione del programma, dove le variabili e i valori vengono definiti prima di essere utilizzati.
 
 Nell'analisi forward, i risultati per ciascun blocco di base dipendono dagli ingressi provenienti dai blocchi predecessori.
\pagebreak
\subsection{Iterazioni dell'algoritmo}
	\begin{longtable}{| c | c | c | c | c | c | c | c | c |}
		\hline
		\textbf{BB} & \multicolumn{2}{|c|}{\textbf{Iterazione 1}} & 	\multicolumn{2}{|c|}{\textbf{iterazione 2}} & 	\multicolumn{2}{|c|}{\textbf{iterazione 3}} & \multicolumn{2}{|c|}{\textbf{iterazione 4}} \\
		\hline
		\space & in & out & in & out & in & out & in & out\\
		\hline
		Entry & $\emptyset$ & $\mu$ & $\emptyset$ & $\emptyset$ & $\emptyset$ & $\emptyset$ & $\emptyset$ & $\emptyset$ \\
		\hline
		1 & $\mu$ & $\mu$ & $\emptyset$ & \parbox{1cm}{\centering\vspace{5pt}(k,2)\vspace{5pt}} & $\emptyset$ & (k,2) & $\emptyset$ & (k,2)\\
		\hline
		2 & $\mu$ & $\mu$ & \parbox{1cm}{\centering\vspace{5pt}(k,2)\vspace{5pt}} & (k,2) & (k,2) & (k,2) & (k,2) & (k,2)\\
		\hline
		3 & $\mu$ & $\mu$ & (k,2) & \parbox{1cm}{\centering\vspace{5pt} (k,2)\\(a,4)\vspace{5pt}} & (k,2) & \parbox{1cm}{\centering (k,2)\\(a,4)} & (k,2)& \parbox{1cm}{\centering (k,2)\\(a,4)} \\
		\hline
		4 & $\mu$ & $\mu$ & \parbox{1cm}{\centering (k,2)\\(a,4)} & \parbox{1cm}{\centering \vspace{5pt}(k,2)\\(a,4) \\ (x,5)\vspace{5pt}} & \parbox{1cm}{\centering (k,2)\\(a,4)}  & \parbox{1cm}{\centering (k,2)\\(a,4)\\ (x,5)}  & \parbox{1cm}{\centering (k,2)\\(a,4)} & \parbox{1cm}{\centering (k,2)\\(a,4)\\ (x,5)} \\
		\hline
		5 & $\mu$ & $\mu$ & (k,2) & \parbox{1cm}{\centering\vspace{5pt} (k,2)\\(a,4)\vspace{5pt}}  & (k,2) & \parbox{1cm}{\centering (k,2)\\(a,4)}  & (k,2)& \parbox{1cm}{\centering (k,2)\\(a,4)} \\
		\hline
		6 & $\mu$ & $\mu$ & \parbox{1cm}{\centering (k,2)\\(a,4)} & \parbox{1cm}{\centering\vspace{5pt} (k,2)\\(a,4)\\ (x,8)\vspace{5pt}}  & \parbox{1cm}{\centering (k,2)\\(a,4)}  & \parbox{1cm}{\centering (k,2)\\(a,4)\\ (x,8)}  &\parbox{1cm}{\centering (k,2)\\(a,4)}  & \parbox{1cm}{\centering (k,2)\\(a,4)\\ (x,8) } \\
		\hline
		7 & $\mu$ & $\mu$ & \parbox{1cm}{\centering (k,2)\\(a,4)}  & \parbox{1cm}{\centering (k,2)\\(a,4)}  & \parbox{1cm}{\centering\vspace{5pt} (k,2)\\(a,4)\vspace{5pt}}  & \parbox{1cm}{\centering (k,2)\\(a,4)}  & \parbox{1cm}{\centering (k,2)\\(a,4)}  & \parbox{1cm}{\centering (k,2)\\(a,4)} \\
		\hline
		8 & $\mu$ & $\mu$ & \parbox{1cm}{\centering\vspace{5pt} (k,2)\\(a,4)\vspace{5pt}}  & \parbox{1cm}{\centering (k,2)\\(a,4)}  & (a,4) & (a,4)  & (a,4)  & (a,4) \\
		\hline
		9 & $\mu$ & $\mu$ & \parbox{1cm}{\centering (k,2)\\(a,4)} & \parbox{1cm}{\centering\vspace{5pt} (b,2)\\(k,2)\\(a,4)\vspace{5pt}} & (a,4) & \parbox{1cm}{\centering (a,4)\\(b,2)} & (a,4) & \parbox{1cm}{\centering (a,4)\\(b,2)}\\
		\hline
		10 & $\mu$ & $\mu$ & \parbox{1cm}{\centering (b,2)\\(k,4)\\ (a,4)} & \parbox{1cm}{\centering\vspace{5pt} (b,2)\\(k,4)\\ (a,4)\\ (x,8)\vspace{5pt}} & \parbox{1cm}{\centering (a,4)\\(b,2)} & \parbox{1cm}{\centering (a,4)\\(b,2)} & \parbox{1cm}{\centering (a,4)\\(b,2)}  & \parbox{1cm}{\centering (a,4)\\(b,2)} \\
		\hline
		11 & $\mu$ & $\mu$ & \parbox{1cm}{\centering \vspace{5pt} (b,2)\\(k,4)\\(a,4)\\(x,8)}  & \parbox{1cm}{\centering \vspace{5pt}  (b,2)\\(k,4)\\ (a,4)\\ (x,8)\\ (y,8)\vspace{5pt}} & \parbox{1cm}{\centering (a,4)\\(b,2)}  & \parbox{1cm}{\centering (a,4)\\(b,2)\\ (y,8)}  & \parbox{1cm}{\centering (a,4)\\(b,2)}  & \parbox{1cm}{\centering (a,4)\\(b,2)\\ (y,8) } \\
		\hline
		12 & $\mu$ & $\mu$ & \parbox{1cm}{\centering (b,2)\\(k,4)\\(a,4)\\(x,8)\\ (y,8)} & \parbox{1cm}{\centering \vspace{5pt} (b,2)\\(a,4)\\(x,8)\\(y,8)\\ (k,5)\vspace{5pt}} & \parbox{1cm}{\centering (a,4)\\ (b,2)\\(y,8)}& \parbox{1cm}{\centering (a,4)\\ (b,2)\\(y,8)} &
		\parbox{1cm}{\centering (a,4)\\ (b,2)\\(y,8)} &\parbox{1cm}{\centering (a,4)\\ (b,2)\\(y,8)}\\
		\hline
		13 & $\mu$ & $\mu$ & 
		\parbox{1cm}{\centering\vspace{5pt} (k,4)\\ (a,4)\vspace{5pt}} &
		\parbox{1cm}{\centering (k,4)\\(a,4)} &
		 (a,4)& (a,4) & (a,4) & (a,4) \\
		\hline
		Exit& $\mu$ & $\mu$ & 
		\parbox{1cm}{\centering\vspace{5pt} (k,4)\\ (a,4)\vspace{5pt}} &
		\parbox{1cm}{\centering (k,4)\\(a,4)} &
		(a,4)& (a,4) & (a,4) & (a,4) \\
		\hline
	\end{longtable}
\pagebreak

	
\end{section}

\end{document}
